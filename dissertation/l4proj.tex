% REMEMBER: You must not plagiarise anything in your report. Be extremely careful.

\documentclass{l4proj}

    
%
% put any additional packages here
%

\begin{document}

%==============================================================================
%% METADATA
\title{Carbon Emissions Estimation in Edge Cloud Computing Simulations}
\author{James A. Nurdin}
\date{September 19, 2023}

\maketitle

%==============================================================================
%% ABSTRACT
\begin{abstract}
    Every abstract follows a similar pattern. Motivate; set aims; describe work; explain results.
    \vskip 0.5em
    ``XYZ is bad. This project investigated ABC to determine if it was better. 
    ABC used XXX and YYY to implement ZZZ. This is particularly interesting as XXX and YYY have
    never been used together. It was found that
    ABC was 20\% better than XYZ, though it caused rabies in half of subjects.''
\end{abstract}

%==============================================================================

% EDUCATION REUSE CONSENT FORM
% If you consent to your project being shown to future students for educational purposes
% then insert your name and the date below to  sign the education use form that appears in the front of the document. 
% You must explicitly give consent if you wish to do so.
% If you sign, your project may be included in the Hall of Fame if it scores particularly highly.
%
% Please note that you are under no obligation to sign 
% this declaration, but doing so would help future students.
%
%\def\consentname {My Name} % your full name
%\def\consentdate {20 March 2018} % the date you agree
%
\educationalconsent


%==============================================================================
\tableofcontents

%==============================================================================
%% Notes on formatting
%==============================================================================
% The first page, abstract and table of contents are numbered using Roman numerals and are not
% included in the page count. 
%
% From now on pages are numbered
% using Arabic numerals. Therefore, immediately after the first call to \chapter we need the call
% \pagenumbering{arabic} and this should be called once only in the document. 
%
% Do not alter the bibliography style.
%
% The first Chapter should then be on page 1. You are allowed 40 pages for a 40 credit project and 30 pages for a 
% 20 credit report. This includes everything numbered in Arabic numerals (excluding front matter) up
% to but excluding the appendices and bibliography.
%
% You must not alter text size (it is currently 10pt) or alter margins or spacing.
%
%
%==================================================================================================================================
%
% IMPORTANT
% The chapter headings here are **suggestions**. You don't have to follow this model if
% it doesn't fit your project. Every project should have an introduction and conclusion,
% however. 
%
%==================================================================================================================================
\chapter{Introduction}

% reset page numbering. Don't remove this!
\pagenumbering{arabic} 


Why should the reader care about what are you doing and what are you actually doing?
\section{Guidance}

\textbf{Motivate} first, then state the general problem clearly. 

\section{Writing guidance}
\subsection{Who is the reader?}


This is the key question for any writing. Your reader:
\begin{itemize}
    \item
    is a trained computer scientist: \emph{don't explain basics}.
    \item
    has limited time: \emph{keep on topic}.
    \item
    has no idea why anyone would want to do this: \emph{motivate clearly}
    \item
    might not know \emph{anything} about your project in particular:
    \emph{explain your project}.
    \item
    but might know precise details and check them: \emph{be precise and
    strive for accuracy.}
    \item
    doesn't know or care about you: \emph{personal discussions are
    irrelevant}.
\end{itemize}

Remember, you will be marked by your supervisor and one or more members
of staff. You might also have your project read by a prize-awarding
committee or possibly a future employer. Bear that in mind.

\subsection{References and style guides}
There are many style guides on good English writing. You don't need to
read these, but they will improve how you write.

\begin{itemize}
    \item
    \emph{How to write a great research paper}~\cite{Pey17} (\textbf{recommended}, even though you aren't writing a research paper)
    \item
    \emph{How to Write with Style} \cite{Von80}. Short and easy to read. Available online.
    \item
    \emph{Style: The Basics of Clarity and Grace} \cite{Wil09} A very popular modern English style guide.
    \item
    \emph{Politics and the English Language} \cite{Orw68}  A famous essay on effective, clear writing in English.
    \item
    \emph{The Elements of Style} \cite{StrWhi07} Outdated, and American, but a classic.
    \item
    \emph{The Sense of Style} \cite{Pin15} Excellent, though quite in-depth.
\end{itemize}

\subsubsection{Citation styles}

\begin{itemize}
\item If you are referring to a reference as a noun, then cite it as: ``\citet{Orw68} discusses the role of language in political thought.''
\item If you are referring implicitly to references, use: ``There are many good books on writing \citep{Orw68, Wil09, Pin15}.''
\end{itemize}

There is a complete guide on good citation practice by Peter Coxhead available here: \url{http://www.cs.bham.ac.uk/~pxc/refs/index.html}. 
If you are unsure about how to cite online sources, please see this guide: \url{https://student.unsw.edu.au/how-do-i-cite-electronic-sources}.

\subsection{Plagiarism warning}

\begin{highlight_title}{WARNING}
    
    If you include material from other sources without full and correct attribution, you are commiting plagiarism. The penalties for plagiarism are severe.
    Quote any included text and cite it correctly. Cite all images, figures, etc. clearly in the caption of the figure.
\end{highlight_title}


%==================================================================================================================================
\chapter{Background}
What did other people do, and how is it relevant to what you want to do?
\section{Guidance}
\begin{itemize}    
    \item
      Don't give a laundry list of references.
    \item
      Tie everything you say to your problem.
    \item
      Present an argument.
    \item Think critically; weigh up the contribution of the background and put it in context.    
    \item
      \textbf{Don't write a tutorial}; provide background and cite
      references for further information.
\end{itemize}

%==================================================================================================================================
\chapter{Analysis/Requirements}\label{ch:analysis/requirements}
The goal of this project is to extend the open-source edge/cloud computing simulator LEAF with
features to configure particular energy sources (with potentially fluctuating availability and variable carbon intensities),
functionality to translate estimates of power consumption of distributed applications and infrastructure components into carbon footprints – would need to to translate energy consumption into greenhouse gas emissions- helpful to
interesting example scenarios that demonstrate the newly added capabilities.
An interesting demonstration could, for example, test mechanisms for distributed software systems that aim to make the most of low-carbon energy over time and locations.

What is the problem that you want to solve, and how did you arrive at it?
\section{Guidance}
Make it clear how you derived the constrained form of your problem via a clear and logical process. 

In order to approach translating estimations in power consumptions into carbon footprints, Extended Leaf needs to
Using the definition provided by \cite{owid-electricity-mix}, we can define carbon intensity as ``the amount of CO2 that is produced per unit of electricity''.
% INTRODUCE NOTATION FOR CARBON INTENSITY Ci
\section{Configuring Energy Sources}
\section{Translating Power Consumption into Carbon Footprints}
mention notation for carbon intensity
mention issues with converting to discrete event space and timings i.e. assume power drawn for delta t is consistent
\section{Demonstrating Scenarios}

goal Attach power to entitites to the infrastructure
from this need a means to go from power consumed to a carbon footprint
introduce a means to show results to the user

- then allow for interactions
%==================================================================================================================================
\chapter{Design}\label{ch:design}

%How is this problem to be approached, without reference to specific implementation details?

%Design should cover the abstract design in such a way that someone else might be able to do what you did, but with a different language or library or tool.

\section{Archtecture Overview}\label{sec:architecture-overview}
% overview
Fundamentally the model operates by separating logic between layers.
At the top layer, the model considers the applications of the simulation which describes the sequential series of tasks and their flow of data from a source task to a sink task.
These applications are then placed on top of the infrastructure layer through a mediation process which identifies the most suitable infrastructure entity to place a task onto.
The infrastructure layer is used to describe the physical compute nodes and network links of the simulation and how these entities are inter-connected.
Extended LEAF sees to introduce a new layer into this architecture by providing a layer underneath the infrastructure who's responsibility relies on distributing power to the nodes and links through user definable power sources.
Similar to how the orchestrator mediates the relationships between the application and infrastructure layer, relations and interactions between power sources and infrastructure depend on power domains as seen in figure \ref{fig:archtecture}.

\begin{figure}[htbp]
    \centering
    \includegraphics[width=0.8\textwidth]{images/Architecture.pdf}
    ~
    \caption{Architecture overview showing interactions between layers.}
    \label{fig:archtecture}
\end{figure}

In general, Extended LEAF was designed in an manner that would compliment the existing simulation framework but not have a mandated requirement for its features to be incorporated to carry out simulations.
As such, the extension was designed to be loosely coupled to the existing framework \citep{looseCoupling}, therefore allowing for existing simulations and scenarios to be performed under no restrictions to realise any power source or power domain.
The framework achieves this by allowing the power layer to have an unrestricted view of the underlying infrastructure of the model, being able to query the state of entities in the above layer and having the power domain carry out necessary actions on the infrastructure.
In particular, the framework operates within a unified discrete event space allowing for a power domain to run simultaneously, as such we can realise interactions between power sources and infrastructure through keeping associations between the two layers respectively.

\section{Power Domains}\label{sec:power-domains}
The main goal of the power domain is to allow for configurations and manage how entities in the infrastructure layer should be distributed amongst power sources.
Power domains should be defined in order to separate the different power options available to a simulation's infrastructure.
For instance when considering power distribution policies for different parts of the infrastructure, individual power domains should be present in order to handle how power sources are allocated, as large scenarios may mean that power sources are present to only part of the infrastructure, see figure 4.2.
\begin{figure}[htbp]
    \centering
    \includegraphics[width=0.7\textwidth]{images/seperatePDDiagram.pdf}
    ~
    \caption{Diagram depicting separate power sources available to parts of the infrastructure.}
    \label{fig:seperatePDs}
\end{figure}

Whilst the framework can carry out simulations without including the new additions of Extended LEAF, if the user wants to provide power sources to the infrastructure, then a power domain must be present to allow for interactions to occur.

\subsection{Workflow}\label{subsec:power-domain-workflow}
In order to ensure that power can be correctly distributed to entities in the infrastructure, the power domain consistently adheres to a predefined workflow as the simulation moves forward in time.
By utilising a discrete event space, Extended LEAF can ensure that the workflow can be completed before time increments in the space.
Therefore, tasks are executed in the following order:
\begin{enumerate}
    \item As time steps forward the first action carried out by the power domain is to execute any defined events allocated by the user \emph{(see section \ref{sec:events}}), this ensures that these actions change the state of the simulation before the normal workflow occurs.
    \item Once any events are executed, the power domain proceeds to update the available power and carbon intensity for all power sources currently allocated to the power domain.
    \item The power domain will then proceed to reorder the power sources based on initial user preference, guaranteeing that the order of power sources remains consistent.
    \item Using this order, each power source determines the active set of infrastructure entities they will provide power to.
    \item After this, the power domain proceeds to calculate how much carbon was produced by each entity and the power sources as a whole.
    \item These measurements are then logged to allow for potential access after the simulation finishes.
    \item Finally, the simulation steps forward in time.
\end{enumerate}

\subsection{Calculating Carbon Emissions}\label{subsec:carbon released}
Another role the power domain takes on is generating estimations in how much carbon was released during the step in time for entities within the infrastructure.
In order to achieve this, the power domain inspects the infrastructure entities present within the power source ($\mathbf{I_{ps}}$) and individually measures the power for each item.
As carbon intensity is defined as the amount of carbon released per kilowatt-hour of energy \citep{owid-electricity-mix}, the power measurement is converted into a discrete amount of energy consumed within the timestep.
This can be defined $\mathbf{Energy_{i}} = \mathbf{Power_{i}} \times 10^{-3} \times \mathbf{\delta T}$, where:
\begin{itemize}
    \item $\mathbf{Energy_{i}}$ is the amount of energy consumed in watt-hours.
    \item $\mathbf{Power_{i}}$ is the power measurement of entity i in watts.
    \item $\mathbf{\delta T}$ is the step in time in hours.
    \item $10^{-3}$ is used to change to the kilo prefix.
\end{itemize}
From this, the power domain can calculate the carbon emitted by finding the product of this and the carbon intensity of the source as $Carbon Released = Energy_{i} \times CI_{ps(t)}$.

\subsection{Recording Measurements}\label{subsec:power-domain-recording-measurements}
The final responsibility of the power domain is to record measurements generated during the simulation.
Once an entity within the infrastructure has had their carbon emissions calculated, the power domain gathers information about the current state of the entity and composes an entry log.
In particular information about the state such as the current time step, the associated power source, the energy consumed (in watt-hours) and the carbon emitted are all associated to the entity.
The power domain then proceeds to store this entry so it can be used later on. \\
As a result of Extended LEAF allowing for power consumption outside the power domain's workflow, the power domain also provides a means to allow for these actions to be logged, ensuring that the file written results of a simulation reflect the events that occurred.

\section{Power Sources}
To allow for advanced scenarios, Extended LEAF allows for a power source to distribute its power in two approaches, statically and dynamically.

idea to use the Liskov Substitution Principle \cite{here} to allow power domains to be interchangeable for infrastructure entities.
- power sources are assumed to fall into one of 4 categories (State them)
- power sources depending on their configuration to always distribute power on update intervals
- due to the discrete nature of the simulation power is calculated in Wh
- available power is treated at a discrete value during a current state and is assumed that every update interval power sources can only distribute what they have been allocated.
- power sources have a heirarchy in which if the user wants to prioritise a particular power source they will give it a high priority
- As LEAF is an estimation simulator power sources draw their data from historical data.
- power sources are allocated a priority to determine what order they should appear in
    $I_{pd}(t) \in I$
~
    $I = I_{pd}^{1}(t) \cup I_{pd}^{2}(t) \cup \ldots \cup I_{pd}^{n}(t) $

\subsection{Static Power Sources}\label{subsec:static-power-sources}
When a power source is defined as static, the user must provide the entities from the infrastructure that they want to associate the source to.
From this, the power source will only distribute power to these provided entities, if a situation occurs where the power source can no longer provide power to these entities then all nodes and links associated will be powered off and any tasks running will be paused until power can be provided to the entity.
The model provides this ability to allow for scenarios when a simulation requires various power sources but needs to impose restrictions on particular entities in the infrastructure.
For instance when considering entities in the infrastructure that are mobile, providing power from a battery is a necessity but is essential that these entities do not receive power from another source in the power domain.\\

\subsection{Dynamic Power Sources}\label{subsec:dynamic-power-sources}
When a power source is defined as dynamic, the user will associate any entities within the infrastructure to the power domain, represented as $\mathbf{I_{pd} \in I}$, where $\mathbf{I}$ encompasses all nodes and links denoted as $\mathbf{\{N\} \cup \{L\}}$.
    From this the responsibility to allocate a power source resolves to the power domains distribution method, whether being the default algorithm provided by Extended LEAF or one defined by the user.
Unlike the statically defined power source configuration, when a scenario occurs where an infrastructure entity is not allocated directly to a power source the simulation fails.\\
When considering entities to allocate power to at time $t$, the default distribution method prioritises nodes into three categories:
\begin{enumerate}
    \item Entities that were previously provided power by the source.
    \item Entities that currently have no power source.
    \item Entities that reside in a less desirable power source.
\end{enumerate}
The distribution method considers infrastructure entities in this order to ensure that nodes and links which have an existing association remain powered before the power source allocates any remaining power to other entities.
In addition to this, the power domain also considers the order in which power sources choose entities in the infrastructure.
As one of the goals of the project is to introduce carbon awareness into the simulation, similar to other systems \cite{cucumber}, the power domain will always distribute infrastructure to the most desirable power sources first.
This will ensure that power sources with a higher carbon intensity at any moment in time, always choose from the smallest set of entities.
This can is formulated as $I_{pd}^j(t) \in I(t)_{r}$ where:
\begin{quote}
    $I_{pd}^j(t)$ is the set of infrastructure entities associated to power domain $j$ at time $t$.\\
    $I(t)_{r}$ is defined as $I \setminus \left( \bigcup_{j-1} I_{pd}^{j-1}(t) \right)$, which describes the entities that are yet to be allocated a power domain.
\end{quote}

\subsection{Power Source Types}\label{subsec:power-source-types}
3 main types:
renewable:
    - static carbon intensity
    - variable power
mixed
    - variable carbon intensity (as a result of combo of non)
    - assumed unlimited power
battery
    - small static carbon intensity (result assumed to come from recharging)
    - finite power
    - power is provided from another source

\subsection{Historical Data}\label{subsec:data Reading}
 - As this is an estimation can rely on historical data to represent trends in both power and carbon intensity
    - type is based on power source type
    - user is able to decide what data is used and can pass their own in

\section{Events}\label{sec:events}
- core feature to integrate functionality and allow for actions/events to occur without going into the power domain/ runner to alter logic
- events can be repeated
- events are called to alter functionality to represent changing conditions of the simulation, allowing for more complex interactions
- events are executed

\subsection{Displaying results}\label{subsec:displaying-results}


- events are logged once power has been consumed
- power consumed
- carbon emitted
- carbon intensity

- further logging is made calculating total carbon emitted for the source in total at time t
- Graphs are produced showing how values change over time
- optional class the user can implement
- passes through the power domain per file writer
- specify what graphs they want to plot
%==================================================================================================================================
\chapter{Implementation}
What did you do to implement this idea, and what technical achievements did you make?
\section{Guidance}
You can't talk about everything. Cover the high level first, then cover important, relevant or impressive details.



\section{General points}

These points apply to the whole dissertation, not just this chapter.



\subsection{Figures}
\emph{Always} refer to figures included, like Figure \ref{fig:relu}, in the body of the text. Include full, explanatory captions and make sure the figures look good on the page.
You may include multiple figures in one float, as in Figure \ref{fig:synthetic}, using \texttt{subcaption}, which is enabled in the template.



% Figures are important. Use them well.
\begin{figure}
    \centering
    \includegraphics[width=0.5\linewidth]{images/relu.pdf}    

    \caption{In figure captions, explain what the reader is looking at: ``A schematic of the rectifying linear unit, where $a$ is the output amplitude,
    $d$ is a configurable dead-zone, and $Z_j$ is the input signal'', as well as why the reader is looking at this: 
    ``It is notable that there is no activation \emph{at all} below 0, which explains our initial results.'' 
    \textbf{Use vector image formats (.pdf) where possible}. Size figures appropriately, and do not make them over-large or too small to read.
    }

    % use the notation fig:name to cross reference a figure
    \label{fig:relu} 
\end{figure}


\begin{figure}
    \centering
    \begin{subfigure}[b]{0.45\textwidth}
        \includegraphics[width=\textwidth]{images/synthetic.png}
        \caption{Synthetic image, black on white.}
        \label{fig:syn1}
    \end{subfigure}
    ~ %add desired spacing between images, e. g. ~, \quad, \qquad, \hfill etc. 
      %(or a blank line to force the subfigure onto a new line)
    \begin{subfigure}[b]{0.45\textwidth}
        \includegraphics[width=\textwidth]{images/synthetic_2.png}
        \caption{Synthetic image, white on black.}
        \label{fig:syn2}
    \end{subfigure}
    ~ %add desired spacing between images, e. g. ~, \quad, \qquad, \hfill etc. 
    %(or a blank line to force the subfigure onto a new line)    
    \caption{Synthetic test images for edge detection algorithms. \subref{fig:syn1} shows various gray levels that require an adaptive algorithm. \subref{fig:syn2}
    shows more challenging edge detection tests that have crossing lines. Fusing these into full segments typically requires algorithms like the Hough transform.
    This is an example of using subfigures, with \texttt{subref}s in the caption.
    }\label{fig:synthetic}
\end{figure}

\clearpage

\subsection{Equations}

Equations should be typeset correctly and precisely. Make sure you get parenthesis sizing correct, and punctuate equations correctly 
(the comma is important and goes \textit{inside} the equation block). Explain any symbols used clearly if not defined earlier. 

For example, we might define:
\begin{equation}
    \hat{f}(\xi) = \frac{1}{2}\left[ \int_{-\infty}^{\infty} f(x) e^{2\pi i x \xi} \right],
\end{equation}    
where $\hat{f}(\xi)$ is the Fourier transform of the time domain signal $f(x)$.

\subsection{Algorithms}
Algorithms can be set using \texttt{algorithm2e}, as in Algorithm \ref{alg:metropolis}.

% NOTE: line ends are denoted by \; in algorithm2e
\begin{algorithm}
    \DontPrintSemicolon
    \KwData{$f_X(x)$, a probability density function returing the density at $x$.\; $\sigma$ a standard deviation specifying the spread of the proposal distribution.\;
    $x_0$, an initial starting condition.}
    \KwResult{$s=[x_1, x_2, \dots, x_n]$, $n$ samples approximately drawn from a distribution with PDF $f_X(x)$.}
    \Begin{
        $s \longleftarrow []$\;
        $p \longleftarrow f_X(x)$\;
        $i \longleftarrow 0$\;
        \While{$i < n$}
        {
            $x^\prime \longleftarrow \mathcal{N}(x, \sigma^2)$\;
            $p^\prime \longleftarrow f_X(x^\prime)$\;
            $a \longleftarrow \frac{p^\prime}{p}$\;
            $r \longleftarrow U(0,1)$\;
            \If{$r<a$}
            {
                $x \longleftarrow x^\prime$\;
                $p \longleftarrow f_X(x)$\;
                $i \longleftarrow i+1$\;
                append $x$ to $s$\;
            }
        }
    }
    
\caption{The Metropolis-Hastings MCMC algorithm for drawing samples from arbitrary probability distributions, 
specialised for normal proposal distributions $q(x^\prime|x) = \mathcal{N}(x, \sigma^2)$. The symmetry of the normal distribution means the acceptance rule takes the simplified form.}\label{alg:metropolis}
\end{algorithm}

\subsection{Tables}

If you need to include tables, like Table \ref{tab:operators}, use a tool like https://www.tablesgenerator.com/ to generate the table as it is
extremely tedious otherwise. 

\begin{table}[]
    \caption{The standard table of operators in Python, along with their functional equivalents from the \texttt{operator} package. Note that table
    captions go above the table, not below. Do not add additional rules/lines to tables. }\label{tab:operators}
    %\tt 
    \rowcolors{2}{}{gray!3}
    \begin{tabular}{@{}lll@{}}
    %\toprule
    \textbf{Operation}    & \textbf{Syntax}                & \textbf{Function}                            \\ %\midrule % optional rule for header
    Addition              & \texttt{a + b}                          & \texttt{add(a, b)}                                    \\
    Concatenation         & \texttt{seq1 + seq2}                    & \texttt{concat(seq1, seq2)}                           \\
    Containment Test      & \texttt{obj in seq}                     & \texttt{contains(seq, obj)}                           \\
    Division              & \texttt{a / b}                          & \texttt{div(a, b) }  \\
    Division              & \texttt{a / b}                          & \texttt{truediv(a, b) } \\
    Division              & \texttt{a // b}                         & \texttt{floordiv(a, b)}                               \\
    Bitwise And           & \texttt{a \& b}                         & \texttt{and\_(a, b)}                                  \\
    Bitwise Exclusive Or  & \texttt{a \textasciicircum b}           & \texttt{xor(a, b)}                                    \\
    Bitwise Inversion     & \texttt{$\sim$a}                        & \texttt{invert(a)}                                    \\
    Bitwise Or            & \texttt{a | b}                          & \texttt{or\_(a, b)}                                   \\
    Exponentiation        & \texttt{a ** b}                         & \texttt{pow(a, b)}                                    \\
    Identity              & \texttt{a is b}                         & \texttt{is\_(a, b)}                                   \\
    Identity              & \texttt{a is not b}                     & \texttt{is\_not(a, b)}                                \\
    Indexed Assignment    & \texttt{obj{[}k{]} = v}                 & \texttt{setitem(obj, k, v)}                           \\
    Indexed Deletion      & \texttt{del obj{[}k{]}}                 & \texttt{delitem(obj, k)}                              \\
    Indexing              & \texttt{obj{[}k{]}}                     & \texttt{getitem(obj, k)}                              \\
    Left Shift            & \texttt{a \textless{}\textless b}       & \texttt{lshift(a, b)}                                 \\
    Modulo                & \texttt{a \% b}                         & \texttt{mod(a, b)}                                    \\
    Multiplication        & \texttt{a * b}                          & \texttt{mul(a, b)}                                    \\
    Negation (Arithmetic) & \texttt{- a}                            & \texttt{neg(a)}                                       \\
    Negation (Logical)    & \texttt{not a}                          & \texttt{not\_(a)}                                     \\
    Positive              & \texttt{+ a}                            & \texttt{pos(a)}                                       \\
    Right Shift           & \texttt{a \textgreater{}\textgreater b} & \texttt{rshift(a, b)}                                 \\
    Sequence Repetition   & \texttt{seq * i}                        & \texttt{repeat(seq, i)}                               \\
    Slice Assignment      & \texttt{seq{[}i:j{]} = values}          & \texttt{setitem(seq, slice(i, j), values)}            \\
    Slice Deletion        & \texttt{del seq{[}i:j{]}}               & \texttt{delitem(seq, slice(i, j))}                    \\
    Slicing               & \texttt{seq{[}i:j{]}}                   & \texttt{getitem(seq, slice(i, j))}                    \\
    String Formatting     & \texttt{s \% obj}                       & \texttt{mod(s, obj)}                                  \\
    Subtraction           & \texttt{a - b}                          & \texttt{sub(a, b)}                                    \\
    Truth Test            & \texttt{obj}                            & \texttt{truth(obj)}                                   \\
    Ordering              & \texttt{a \textless b}                  & \texttt{lt(a, b)}                                     \\
    Ordering              & \texttt{a \textless{}= b}               & \texttt{le(a, b)}                                     \\
    % \bottomrule
    \end{tabular}
    \end{table}
\subsection{Code}

Avoid putting large blocks of code in the report (more than a page in one block, for example). Use syntax highlighting if possible, as in Listing \ref{lst:callahan}.

\begin{lstlisting}[language=python, float, caption={The algorithm for packing the $3\times 3$ outer-totalistic binary CA successor rule into a 
    $16\times 16\times 16\times 16$ 4 bit lookup table, running an equivalent, notionally 16-state $2\times 2$ CA.}, label=lst:callahan]
    def create_callahan_table(rule="b3s23"):
        """Generate the lookup table for the cells."""        
        s_table = np.zeros((16, 16, 16, 16), dtype=np.uint8)
        birth, survive = parse_rule(rule)

        # generate all 16 bit strings
        for iv in range(65536):
            bv = [(iv >> z) & 1 for z in range(16)]
            a, b, c, d, e, f, g, h, i, j, k, l, m, n, o, p = bv

            # compute next state of the inner 2x2
            nw = apply_rule(f, a, b, c, e, g, i, j, k)
            ne = apply_rule(g, b, c, d, f, h, j, k, l)
            sw = apply_rule(j, e, f, g, i, k, m, n, o)
            se = apply_rule(k, f, g, h, j, l, n, o, p)

            # compute the index of this 4x4
            nw_code = a | (b << 1) | (e << 2) | (f << 3)
            ne_code = c | (d << 1) | (g << 2) | (h << 3)
            sw_code = i | (j << 1) | (m << 2) | (n << 3)
            se_code = k | (l << 1) | (o << 2) | (p << 3)

            # compute the state for the 2x2
            next_code = nw | (ne << 1) | (sw << 2) | (se << 3)

            # get the 4x4 index, and write into the table
            s_table[nw_code, ne_code, sw_code, se_code] = next_code

        return s_table

\end{lstlisting}

%==================================================================================================================================
\chapter{Evaluation} 
How good is your solution? How well did you solve the general problem, and what evidence do you have to support that?

\section{Guidance}
\begin{itemize}
    \item
        Ask specific questions that address the general problem.
    \item
        Answer them with precise evidence (graphs, numbers, statistical
        analysis, qualitative analysis).
    \item
        Be fair and be scientific.
    \item
        The key thing is to show that you know how to evaluate your work, not
        that your work is the most amazing product ever.
\end{itemize}

\section{Evidence}
Make sure you present your evidence well. Use appropriate visualisations, reporting techniques and statistical analysis, as appropriate.

If you visualise, follow the basic rules, as illustrated in Figure \ref{fig:boxplot}:
\begin{itemize}
\item Label everything correctly (axis, title, units).
\item Caption thoroughly.
\item Reference in text.
\item \textbf{Include appropriate display of uncertainty (e.g. error bars, Box plot)}
\item Minimize clutter.
\end{itemize}

See the file \texttt{guide\_to\_visualising.pdf} for further information and guidance.

\begin{figure}
    \centering
    \includegraphics[width=1.0\linewidth]{images/boxplot_finger_distance.pdf}    

    \caption{Average number of fingers detected by the touch sensor at different heights above the surface, averaged over all gestures. Dashed lines indicate
    the true number of fingers present. The Box plots include bootstrapped uncertainty notches for the median. It is clear that the device is biased toward 
    undercounting fingers, particularly at higher $z$ distances.
    }

    % use the notation fig:name to cross reference a figure
    \label{fig:boxplot} 
\end{figure}


%==================================================================================================================================
\chapter{Conclusion}    
Summarise the whole project for a lazy reader who didn't read the rest (e.g. a prize-awarding committee).
\section{Guidance}
\begin{itemize}
    \item
        Summarise briefly and fairly.
    \item
        You should be addressing the general problem you introduced in the
        Introduction.        
    \item
        Include summary of concrete results (``the new compiler ran 2x
        faster'')
    \item
        Indicate what future work could be done, but remember: \textbf{you
        won't get credit for things you haven't done}.
\end{itemize}

%==================================================================================================================================
%
% 
%==================================================================================================================================
%  APPENDICES  

\begin{appendices}

\chapter{Appendices}

Typical inclusions in the appendices are:

\begin{itemize}
\item
  Copies of ethics approvals (required if obtained)
\item
  Copies of questionnaires etc. used to gather data from subjects.
\item
  Extensive tables or figures that are too bulky to fit in the main body of
  the report, particularly ones that are repetitive and summarised in the body.

\item Outline of the source code (e.g. directory structure), or other architecture documentation like class diagrams.

\item User manuals, and any guides to starting/running the software.

\end{itemize}

\textbf{Don't include your source code in the appendices}. It will be
submitted separately.

\end{appendices}

%==================================================================================================================================
%   BIBLIOGRAPHY   

% The bibliography style is abbrvnat
% The bibliography always appears last, after the appendices.

\bibliographystyle{abbrvnat}

\bibliography{l4proj}

\end{document}
