    
\documentclass[11pt]{article}
\usepackage{times}
    \usepackage{fullpage}
\usepackage{enumitem}
    
    \title{Carbon Emissions Estimation in Edge Cloud Computing Simulations Status Report}
    \author{James Nurdin 2570809n}

    \begin{document}

    \maketitle


\section{Status report}

\subsection{Proposal}\label{proposal}

\subsubsection{Motivation}\label{motivation}

As the information and communications industry is increasingly consuming larger quantities of the global energy demand,
a possible solution to the problem exists in the form of a distributed computing process called fog computing.
A simulation tool called LEAF was previously developed in order to provide a graph based approach to model Large Energy-Aware Fog
computing environments in order to facilitate research and industrial applications of the technique.
With sustainable computing becoming a major subfield within the discipline of computer science the ability to augment the
framework to allow for simulations to be carbon aware could provide further research and industrial opportunities to utilise fog computing.
By allowing the model to have a carbon awareness during the runtime of the simulation, the framework allows the user to
exploit periods of low carbon intensity or potential onsite excess power to optimise the execution of tasks for
compute nodes to minimise the models carbon footprint.

\subsubsection{Aims}\label{aims}

This project aims to extend the capabilities of LEAF in order to provide users with the ability to estimate potential environmental footprints
of simulations through the quantity of carbon being released.
In particular the extension aims to provide a suite of features to allow users to curate custom power sources and the ability to control
how they are distributed amongst a defined infrastructure.
The framework will also provide the ability to explicitly define conditions relating to the nature of the simulation
such as the environment and restrictions of the infrastructure to allow for the accurate capture of carbon emissions.
The result of this will generate a realistic report of how carbon is released during the simulation.
The effectiveness of the project will be evaluating various scenarios to:
First demonstrate the accuracy of estimated carbon footprints of
smaller scenarios.
Then investigate and analyse realistic examples to demonstrate how the distribution of computational nodes amongst
aggregated power sources and environmental conditions can have adverse affects on the total carbon emission, and how
exploiting opportunities can reduce the carbon footprint.

\newpage

\subsection{Progress}\label{progress}

\begin{itemize}[itemsep=0pt]
  \item Initial reading material and research carried out, along with other relevant literature.
  \item Using the existing framework chose and pulled the repository chosen to base extended version upon.
  \item Learned the existing framework's inner workings and creating small examples to demonstrate understanding.
  \item Extracted relevant files to build upon and created the initial version of the extended framework.
  \item Implemented the core features of the framework to enable carbon emissions to be generated.
  \item Researched and incorporated relevant online datasets to allow for accurate demonstrations.
  \item Created small examples to demonstrate the core aspects of the new framework by including default implementations.
  \item Started research and development of main example to be used in dissertation to evaluate the effectiveness of the framework.
\end{itemize}

\subsection{Problems and risks}\label{problems-and-risks}

\subsubsection{Problems}\label{problems}

\begin{itemize}[itemsep=0pt]
\item Assumptions concerning power sources limiting the scope of realistic scenarios.
\item Choosing initial data sets to base default power sources from that were appropriate and available to use.
\item Deciding on an appropriate final example to demonstrate the framework that is realistic but can be developed using the framework.
\item Ensuring that carbon emissions were correct and accurate.
\item Various refactors to the framework to improve usability.
\item Utilising time based calculations whilst working within a discrete event space.
\end{itemize}

\subsubsection{Risks}\label{risks}

\begin{itemize}[itemsep=0pt]
    \item Core features of the framework are either finished or are close, however a collection of smaller features are still available to be included but the main example is currently taking priority.
        \textbf{Mitigation}: Concurrently work on the core example while introducing features along side development prioritising worthwhile features first and omitting smaller features if unable to incorporate them.
    \item The main example will require realistic data and a strong understanding of the scenario and context respectively.
        \textbf{Mitigation}: Research the context of the scenario and ensure that accurate data is used and the example is capable of demonstrating the extended framework.
    \item While carbon emissions are calculated currently no approach to permanently capture them outside of simulations is being used.
        \textbf{Mitigation}: Capture and record all quantities generated as raw data and proceed to allow for optional processing into figures as an option.
\end{itemize}

\newpage

\subsection{Plan}\label{plan}
\subsubsection{Semester 1 and December period:}\label{Semester 1}
    For the remaining weeks before the start of semester 2 the time will be spent working on the agreed aspects discussed during the meeting on the 6th of December 2023.
    In particular this will cover reworking the main example's context, exploring potential figures that will be used
    during the evaluation of the product and incorporating the remaining core features and introducing a testing suite into the project to ensure the framework's correctness during development.
\subsubsection{Semester 2:}\label{Semester 2}
\begin{description}
  \item[Week 1:] Ensure that all core features of the framework are completed along with accompanying test suites.
    \begin{itemize}
      \item Deliverable: A complete framework along with a test suite to validate functionality of extended features.
    \end{itemize}
\end{description}
\begin{description}
  \item[Week 2:] Ensure that all supplementary examples are finished and clearly demonstrate the framework's capabilities.
    \begin{itemize}
      \item Deliverable: A directory of executable examples demonstrating functionality and is capable of teaching the framework to new users.
    \end{itemize}
\end{description}
\begin{description}
  \item[Week 3-4:] Ensure the code for the main example is finalised in preparation for the evaluation.
    \begin{itemize}
      \item Deliverable: In a separate directory to the main project, an isolated example exists capable of various states to evaluate along with the ability to generate results and figures.
    \end{itemize}
\end{description}
\begin{description}
  \item[Week 5:] Draft and plan the dissertation and include roughly what each core section will include.
    \begin{itemize}
      \item Deliverable: A short document to be presented during the weekly meeting showing the approach and potential updated plan for the dissertation.
    \end{itemize}
\end{description}
\begin{description}
  \item[Week 6:] Conduct the evaluation of the project.
    \begin{itemize}
      \item Deliverable: Quantitative and qualitative results showing how different configurations and techniques of the framework can be used to adjust the carbon footprint of the system.
    \end{itemize}
\end{description}
    \begin{description}
  \item[Week 7-8:] Carry out the first write up of the dissertation.
    \begin{itemize}
      \item Deliverable: A first draft submitted to supervisor to get initial feedback before the deadline.
    \end{itemize}
\end{description}
    \begin{description}
  \item[Week 9-10:] Carry out necessary changes and prepare for final submission.
    \begin{itemize}
      \item Deliverable: A polished and refined source code, along with necessary files and changes of the dissertation ready to be submitted on time.
    \end{itemize}
\end{description}
\subsection{Ethics and data}\label{ethics}

After reading the checklist for ethics, this project does not involve human subjects or data.
Therefore no approval is required for me to request.

\end{document}
